% general page/margin sizes

\setlength{\oddsidemargin}{-.25cm}
\setlength{\evensidemargin}{-.25cm}
\setlength{\topmargin}{-50pt}
\setlength{\headheight}{1.5cm}
\setlength{\textwidth}{16cm}
\setlength{\textheight}{23cm}

% spacing lengths (note that we should define our own itemize environment
% customizing the sizes with the second argument {decls}
% cf. p112 of the LaTeX book or p59-62 of the Companion book

\setlength{\parindent}{0cm}
\setlength{\parskip}{\baselineskip}
\setlength{\partopsep}{-\baselineskip}
\setlength{\topsep}{0pt}

	% save parskip for destroy+restore (cf. cover, tbl-contents)

\newlength{\saveparskip}
\setlength{\saveparskip}{\parskip}


% To avoid underfull vbox errors due to [twosides]

\raggedbottom


% Fancy headings

\pagestyle{fancy}
%\setlength{\headrulewidth}{0.8pt}
\renewcommand{\headrulewidth}{0.8pt}

\lhead[\thepage]{\rightmark}
\chead{}
\rhead[\leftmark]{\thepage}
\lfoot{}
\cfoot{}
\rfoot{}



% New space for subsubsection numbers in the table of contents

%BEGIN LATEX
\makeatletter
\renewcommand{\l@subsubsection}{\@dottedtocline{3}{3.8em}{3.6em}}
\makeatother
%END LATEX



% Vertical space commands

\newcommand{\BL}{\vspace{\baselineskip}}
\newcommand{\SkipUp}{\vspace{-\multicolsep}}


% Some characters in tt font

\def\bs{\char'134}
\def\lt{\char'074}
\def\gt{\char'076}
\def\lb{\char'173}
\def\rb{\char'175}
\def\us{\char'137}


% Style of a parameter

\def\Param#1{\texttt{\textit{#1}}}



% A reference to a section/page

\newcommand{\RefSP}[1]{(section~\ref{#1}, page~\pageref{#1})}


% Url in LaTeX output

\ifpdf
\newcommand{\Tilde}[1]{~#1}
\newcommand{\MyUrl}[2]{\href{#1}{#2}}
\newcommand{\MyUrlHtml}[2]{\href{#1}{#2}}
\else
\newcommand{\Tilde}[1]{\~{}#1}
\newcommand{\MyUrl}[2]{\footahref{#1}{#2}}
\newcommand{\MyUrlHtml}[2]{#2}
\fi
\newcommand{\MyEMail}[2]{#2\footnote{\texttt{#1}}}

% General environments

\newenvironment{CmdOptions}%
   {\begin{tabular}{p{4.5cm}l}}%
   {\end{tabular}}


\newenvironment{ItemizeThreeCols}%
   {\begin{multicols}{3}\raggedcolumns\begin{itemize}}%
   {\end{itemize}\end{multicols}\SkipUp}

\newenvironment{Indentation}%
   {\begin{list}{}{}%
      \item }%
   {\end{list}}

\newenvironment{Code}%
   {\begin{Indentation}\begin{tt}}%
   {\end{tt}\end{Indentation}}

\newenvironment{CodeTwoCols}[1][4cm]%
   {\begin{Indentation}\begin{tabular}{@{}p{#1}@{}l@{}}}%
   {\end{tabular}\end{Indentation}}

\def\One#1{\multicolumn{2}{@{}l}{\texttt{#1}} \\}
\def\Two#1#2{\texttt{#1} & #2\\}
\def\SkipLine{\multicolumn{2}{@{}l}{} \\}
\def\OneLine#1{\begin{Code}#1\end{Code}}
\newcommand{\OneLineTwoCols}[3][4cm]%
   {\begin{CodeTwoCols}[#1]\Two{#2}{#3}\end{CodeTwoCols}}

% fix for HeVeA info mode (\\ on a single line causes HeVeA to loop)
%
%\renewcommand{\OneLineTwoCols}[3][4cm]%
%   {\begin{CodeTwoCols}[#1]\texttt{#2} & {#3}\end{CodeTwoCols}}



\newlength{\tmplg}
\newcounter{colnbround}
\newenvironment{TabularC}[1]%
   {\setcounter{colnbround}{#1}
    \addtocounter{colnbround}{1}
    \setlength{\tmplg}{\linewidth/#1 - \tabcolsep*2 %
                       - \arrayrulewidth*\value{colnbround}/#1}%
    \par\begin{tabular*}{\linewidth}%
                      {|*{#1}{>{\raggedright\arraybackslash\hspace{0pt}}%
                         m{\the\tmplg}|}}}%
   {\end{tabular*}\par}


% For use in tabular (parameter is column width)

\newcolumntype{C}[1]{>{\centering\arraybackslash}m{#1}}
\newcolumntype{L}[1]{>{\raggedright\arraybackslash}m{#1}}
\newcolumntype{R}[1]{>{\raggedleft\arraybackslash}m{#1}}



% Image inclusion

\newcommand{\InsertImage}[2][scale=0.83]%
   {\BL\begin{center}\includegraphics[#1]{#2}\end{center}\BL}



% Bips description

\def\SPart#1{\textbf{#1}}
\def\Templates{\SPart{Templates}}
\def\Description{\SPart{Description}}
\def\Errors{\SPart{Errors}}
\def\Portability{\SPart{Portability}}

\newenvironment{TemplatesOneCol}%
   {\Templates\par\begin{Code}}%
   {\end{Code}}


\newenvironment{TemplatesTwoCols}%
   {\Templates\par\begin{multicols}{2}\raggedcolumns\begin{Code}}%
   {\end{Code}\end{multicols}\SkipUp}


\def\PlErrorsNone{\Errors\par None.}

\newenvironment{PlErrorsNoTitle}%
   {\par\begin{TabularC}{2}\hline}%
   {\end{TabularC}}

\newenvironment{PlErrors}%
   {\Errors\begin{PlErrorsNoTitle}}%
   {\end{PlErrorsNoTitle}}

\def\ErrCond#1{#1 &}
\def\ErrTerm#1{\texttt{#1} \\ \hline}
\def\ErrTermRm#1{#1 \\ \hline}

% Index

   % new environnement to use \pagestyle{fancy} and \section{}
   % copied from article.cls with the following changes:
   % remove \thispagestyle{plain}
   % add \addcontentsline{toc}{section}{\numberline{}\indexname}}

%BEGIN LATEX
\makeatletter
\renewenvironment{theindex}
               {\if@twocolumn
                  \@restonecolfalse
                \else
                  \@restonecoltrue
                \fi
                \columnseprule \z@
                \columnsep 35\p@
                \twocolumn[\section*{\indexname}]%
                \@mkboth{\MakeUppercase\indexname}%
                        {\MakeUppercase\indexname}%
                \parindent\z@
                \parskip\z@ \@plus .3\p@\relax
		\addcontentsline{toc}{section}{\numberline{}\indexname}
                \let\item\@idxitem}
               {\if@restonecol\onecolumn\else\clearpage\fi}
\makeatother
%END LATEX

\ifpdf
\def\OneUrl#1{\href{#1}{#1}}
\else
\def\OneUrl#1{\ahrefurl{#1}}
\fi


% Define a section without no and include it in the TOC
% 1= a label (name) for HeVeA (useless for LaTeX)
% 2= The text of the section
% See also redef for HeVeA in custom.hva

\newcommand{\SectionWithoutNo}[2]
{\section*{#2}%
\addcontentsline{toc}{section}{\numberline{}#2}}



% Index management:
% in the following, suffix T/TD/D/void means:
% T: texttt, D: definition, TD: both, void: simple


   % |textbf for \index does not work with hyperref (pdflatex)...
   % We need to pass by the below 'IndexBold' command.
\ifpdf
   \newcommand{\IndexBold}[1]{\textbf{\hyperpage{#1}}}
\else
   \newcommand{\IndexBold}[1]{\textbf{#1}}
\fi


   % Add an index entry
   % 1=alphabetic position  2=complete index term

\newcommand{\IndT} [2]{\index{#1@\texttt{#2}}}
\newcommand{\IndTD}[2]{\index{#1@\texttt{#2}|IndexBold}}
\newcommand{\IndD} [2]{\index{#1@#2|IndexBold}}
\newcommand{\Ind}  [2]{\index{#1@#2}}


   % Add text and an index entry
   % 1=text term  2=alphabetic position  3=complete index term
   % NB: define the \index before the text to have correct HTML anchors

\newcommand{\TxtIndT} [3]{\IndT{#2}{#3}\texttt{#1}}
\newcommand{\TxtIndTD}[3]{\IndTD{#2}{#3}\texttt{#1}}
\newcommand{\TxtIndD} [3]{\IndD{#2}{#3}#1}
\newcommand{\TxtInd}  [3]{\Ind{#2}{#3}#1}


   % Add... macros insert something in the index ONLY
   % Idx... macros insert something in the index AND in the text

   % Any word (roman font)


\newcommand{\AddD}  [1]{\IndD    {#1}{#1}}
\newcommand{\Add}   [1]{\Ind     {#1}{#1}}
\newcommand{\IdxD}  [1]{\TxtIndD {#1}{#1}{#1}}
\newcommand{\Idx}   [1]{\TxtInd  {#1}{#1}{#1}}


   % Keyword (tt font)


\newcommand{\AddKD} [1]{\IndTD   {#1}{#1}}
\newcommand{\AddK}  [1]{\IndT    {#1}{#1}}
\newcommand{\IdxKD} [1]{\TxtIndTD{#1}{#1}{#1}}
\newcommand{\IdxK}  [1]{\TxtIndT {#1}{#1}{#1}}


   % Directive


\newcommand{\AddDiD}[1]{\IndTD   {#1}{#1 \textrm{(directive)}}}
\newcommand{\AddDi} [1]{\IndT    {#1}{#1 \textrm{(directive)}}}
\newcommand{\IdxDiD}[1]{\TxtIndTD{#1}{#1}{#1 \textrm{(directive)}}}
\newcommand{\IdxDi} [1]{\TxtIndT {#1}{#1}{#1 \textrm{(directive)}}}


   % Control Construct


\newcommand{\AddCCD}[1]{\IndTD   {#1}{#1}}
\newcommand{\AddCC} [1]{\IndT    {#1}{#1}}
\newcommand{\IdxCCD}[1]{\TxtIndTD{#1}{#1}{#1}}
\newcommand{\IdxCC} [1]{\TxtIndT {#1}{#1}{#1}}


   % Prolog Keyword


\newcommand{\AddPKD}[1]{\IndTD   {#1}{#1}}
\newcommand{\AddPK} [1]{\IndT    {#1}{#1}}
\newcommand{\IdxPKD}[1]{\TxtIndTD{#1}{#1}{#1}}
\newcommand{\IdxPK} [1]{\TxtIndT {#1}{#1}{#1}}


   % Prolog Built-in


\newcommand{\AddPBD}[1]{\IndTD   {#1}{#1}}
\newcommand{\AddPB} [1]{\IndT    {#1}{#1}}
\newcommand{\IdxPBD}[1]{\TxtIndTD{#1}{#1}{#1}}
\newcommand{\IdxPB} [1]{\TxtIndT {#1}{#1}{#1}}


   % Prolog Property


\newcommand{\AddPPD}[1]{\IndTD   {#1}{#1 \textrm{(property)}}}
\newcommand{\AddPP} [1]{\IndT    {#1}{#1 \textrm{(property)}}}
\newcommand{\IdxPPD}[1]{\TxtIndTD{#1}{#1}{#1 \textrm{(property)}}}
\newcommand{\IdxPP} [1]{\TxtIndT {#1}{#1}{#1 \textrm{(property)}}}


   % Prolog Global Variable


\newcommand{\AddPGD}[1]{\IndTD   {#1}{#1 \textrm{(global var.)}}}
\newcommand{\IdxPGD}[1]{\TxtIndTD{#1}{#1}{#1 \textrm{(global var.)}}}


   % Prolog Option


\newcommand{\AddPOD}[1]{\IndTD   {#1}{#1 \textrm{(option)}}}
\newcommand{\AddPO} [1]{\IndT    {#1}{#1 \textrm{(option)}}}
\newcommand{\IdxPOD}[1]{\TxtIndTD{#1}{#1}{#1 \textrm{(option)}}}
\newcommand{\IdxPO} [1]{\TxtIndT {#1}{#1}{#1 \textrm{(option)}}}


   % Prolog Mode


\newcommand{\AddPMD}[1]{\IndTD   {#1}{#1 \textrm{(mode)}}}
\newcommand{\IdxPMD}[1]{\TxtIndTD{#1}{#1}{#1 \textrm{(mode)}}}


   % Prolog Whence


\newcommand{\AddPWD}[1]{\IndTD   {#1}{#1 \textrm{(whence)}}}
\newcommand{\IdxPWD}[1]{\TxtIndTD{#1}{#1}{#1 \textrm{(whence)}}}


   % Prolog File Permission


\newcommand{\AddPXD}[1]{\IndTD   {#1}{#1 \textrm{(permission)}}}
\newcommand{\IdxPXD}[1]{\TxtIndTD{#1}{#1}{#1 \textrm{(permission)}}}


   % Prolog Token


\newcommand{\AddPTD}[1]{\IndTD   {#1}{#1 \textrm{(token)}}}
\newcommand{\IdxPTD}[1]{\TxtIndTD{#1}{#1}{#1 \textrm{(token)}}}


   % Prolog Flag


\newcommand{\AddPFD}[1]{\IndTD   {#1}{#1 \textrm{(flag)}}}
\newcommand{\AddPF} [1]{\IndT    {#1}{#1 \textrm{(flag)}}}
\newcommand{\IdxPFD}[1]{\TxtIndTD{#1}{#1}{#1 \textrm{(flag)}}}
\newcommand{\IdxPF} [1]{\TxtIndT {#1}{#1}{#1 \textrm{(flag)}}}


   % Debugger Keyword


\newcommand{\AddDKD}[1]{\IndTD   {#1}{#1 \textrm{(debug)}}}
\newcommand{\AddDK} [1]{\IndT    {#1}{#1 \textrm{(debug)}}}
\newcommand{\IdxDKD}[1]{\TxtIndTD{#1}{#1}{#1 \textrm{(debug)}}}
\newcommand{\IdxDK} [1]{\TxtIndT {#1}{#1}{#1 \textrm{(debug)}}}


   % Debugger Built-in


\newcommand{\AddDBD}[1]{\IndTD   {#1}{#1 \textrm{(debug)}}}
\newcommand{\AddDB} [1]{\IndT    {#1}{#1 \textrm{(debug)}}}
\newcommand{\IdxDBD}[1]{\TxtIndTD{#1}{#1}{#1 \textrm{(debug)}}}
\newcommand{\IdxDB} [1]{\TxtIndT {#1}{#1}{#1 \textrm{(debug)}}}


   % FD Built-in


\newcommand{\AddFBD}[1]{\IndTD   {#1}{#1 \textrm{(FD)}}}
\newcommand{\AddFB} [1]{\IndT    {#1}{#1 \textrm{(FD)}}}
\newcommand{\IdxFBD}[1]{\TxtIndTD{#1}{#1}{#1 \textrm{(FD)}}}
\newcommand{\IdxFB} [1]{\TxtIndT {#1}{#1}{#1 \textrm{(FD)}}}


   % FD Option


\newcommand{\AddFOD}[1]{\IndTD   {#1}{#1 \textrm{(FD option)}}}
\newcommand{\IdxFOD}[1]{\TxtIndTD{#1}{#1}{#1 \textrm{(FD option)}}}


   % FD Keyword


\newcommand{\AddFKD}[1]{\IndTD   {#1}{#1 \textrm{(FD)}}}
\newcommand{\AddFK} [1]{\IndT    {#1}{#1 \textrm{(FD)}}}
\newcommand{\IdxFKD}[1]{\TxtIndTD{#1}{#1}{#1 \textrm{(FD)}}}
\newcommand{\IdxFK} [1]{\TxtIndT {#1}{#1}{#1 \textrm{(FD)}}}
